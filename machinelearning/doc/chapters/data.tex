\section{Random Numbers in OpenCV}
Each thread in OpenCV has access to a default random number generator \lstinline|cv::theRNG()|. For a single pass the seed for the random number generator doesn't have to be set, but if many random numbers have to be generated (for example in a loop) the seed has to be set.

This can be done by assigning the local time to the random number generator:

\begin{lstlisting}
 cv::theRNG() = cv::RNG(time(0))
\end{lstlisting}

\subsection{Uniform Distribution}
Uniform Distributions can be generated in OpenCV with the help of \textit{cv::randu}. The signature of \textit{cv::randu} is:
\begin{lstlisting}[language=C++]
void randu(Mat& mtx, const Scalar& low, const Scalar& high);
\end{lstlisting}
Where
\begin{itemize}
 \item \textbf{mtx} is the Matrix to be filled with uniform distributed random numbers
 \item \textbf{low} is the inclusive lower boundary of generated random numbers
 \item \textbf{high} is the exclusive upper boundary of generated random numbers
\end{itemize}


\subsection{Normal Distribution}
Normal Distribution can be generated in OpenCv with the help of \textit{cv::randn}.
\begin{lstlisting}[language=C++]
void randn(Mat& mtx, const Scalar& mean, const Scalar& stddev);
\end{lstlisting}
Where
\begin{itemize}
 \item \textbf{mtx} is the Matrix to be filled with normal distributed random numbers
 \item \textbf{mean} the mean value of the generated random numbers
 \item \textbf{stddev} the standard deviation of the generated random numbers
\end{itemize}

\subsection{Bivariate Gaussian Distribution}

Multivariate normal distributions can be generated with the C API by using \textit{cvRandMVNormal} from the Machine Learning library. However this function seems to have no C++ equivalent yet, so in order to generate random numbers for the bivariate case one can also use the function given in \ref{lst:bivar}.

\begin{lstlisting}[language=C++, caption=bivariate\_gaussian dstribution, label=lst:bivar]
void bivariate_gaussian(float mu_x, float sigma_x, float mu_y, float sigma_y, float rho, cv::Mat &x, cv::Mat &y) {
  assert(x.rows == y.rows && x.rows > 0);

  int n = x.rows;

  randn(x, mu_x, sigma_x);

  cv::Mat r(n,1, CV_32FC1);
  cv::theRNG() = cv::RNG(time(0));

  randn(r, 0, 1);
  for(int i = 0; i < n; i++) {
    y.at<float>(i, 0) = sqrt(sigma_y*sigma_y - (rho * sigma_y)*(rho * sigma_y)) * r.at<float>(i, 0) + mu_y;
    y.at<float>(i, 0) = y.at<float>(i,0) + rho * sigma_y/sigma_x * (x.at<float>(i,0) - mu_x);
}
\end{lstlisting}


\section{Preparing the Training and Test Dataset for OpenCV ML}
In the C++ Machine Learning API of OpenCV training and test data is given as a \lstinline|cv::Mat| matrix. The constructor of \lstinline|cv::Mat| is defined as:
\begin{lstlisting}[language=C++]
Mat::Mat(int rows, int cols, int type);
\end{lstlisting}
Where
\begin{itemize}
 \item rows is the number of samples (for all classes!)
 \item columns is the number of dimensions
 \item type is the image type
\end{itemize}

In the machine learning library of OpenCV each row or column in the training data is a n-dimensional sample. The default ordering is row sampling and class labels are given in a matrix with equal length (one column only, of course). 

\begin{lstlisting}[language=C++]
cv::Mat trainingData(numTrainingPoints, 2, CV_32FC1);
cv::Mat testData(numTestPoints, 2, CV_32FC1);

cv::randu(trainingData,0,1);
cv::randu(testData,0,1);

cv::Mat trainingClasses = labelData(trainingData, equation);
cv::Mat testClasses = labelData(testData, equation);
\end{lstlisting}

Since only binary classification problems are considered the function \lstinline|f| returns the classes $-1$ and $1$ for a given two-dimensional data point:

\begin{lstlisting}[language=C++]
// function to learn
int f(float x, float y, int equation) {
  switch(equation) {
  case 0:
    return y > sin(x*10) ? -1 : 1;
    break;
  case 1:
    return y > cos(x * 10) ? -1 : 1;
    break;
  case 2:
    return y > 2*x ? -1 : 1;
    break;
  case 3:
    return y > tan(x*10) ? -1 : 1;
    break;
  default:
    return y > cos(x*10) ? -1 : 1;
  }
}
\end{lstlisting}

And to label data one can use the function \lstinline|labelData|:
\begin{lstlisting}
cv::Mat labelData(cv::Mat points, int equation) {
  cv::Mat labels(points.rows, 1, CV_32FC1);
  for(int i = 0; i < points.rows; i++) {
    float x = points.at<float>(i,0);
    float y = points.at<float>(i,1);
    labels.at<float>(i, 0) = f(x, y, equation);
  }
  return labels;
}
\end{lstlisting}